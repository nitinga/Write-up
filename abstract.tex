Matrix completion, is the task of filling in the unknown entries of a sparse matrix. This problem is central in recommender systems and collaborative filtering, which provide individualized product suggestions based on ratings gathered from a large set of users. Matrix completion algorithms are used effectively today to provide helpful recommendations for users of services such as Netflix and Amazon.com. In the present paper, the baseline algorithms for matrix completion, such as eigentaste, nearest K-neighbor, and column mean are compared against the more sophisticated matrix completion algorithms, such as Spectral Matrix Completion (SMC) and Singular Value Thresholding (SVT). The comparison metric used in the present paper include, accuracy, convergence time, and the nature of the input matrix. All of the presented methods have their advantages and disadvantages based on the problem. In the present paper, the detailed results are presented for synthetic data and the Jester data. While Eigentaste needs at least few fully filled columns to complete the sparse matrix, SMC and SVT does not have that limitation. Between SMC and SVT, SMC provides better accuracy but is much slower than the SVT. Detailed comparison of the methods is presented in the later sections.
