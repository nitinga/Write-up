\documentclass{article} % For LaTeX2e
\usepackage{nips15submit_e,times}
\usepackage{hyperref}
\usepackage{url}
\usepackage{amsmath}
\usepackage{graphicx}
\usepackage{multicol, multirow, array}

\newcommand{\mc}[2]{\multicolumn{#1}{#2}}

%\documentstyle[nips14submit_09,times,art10]{article} % For LaTeX 2.09

\title{Singular Value Thresholding for Matrix Completion}

\author{
Sriram Ganesan\thanks{ The names are printed in alphabetical order by last name.} \\
\texttt{sriramg@umich.edu} \\
\And
Nitin Garg \\
\texttt{gargn@umich.edu} \\
\AND
Jeeheh Oh \\
\texttt{jeeheh@umich.edu} \\
\And
Jonathan Stroud \\
\texttt{stroud@umich.edu} \\
}

\newcommand{\fix}{\marginpar{FIX}}
\newcommand{\new}{\marginpar{NEW}}

\nipsfinalcopy % Uncomment for camera-ready version

\begin{document}

\maketitle

\begin{abstract}
Recommender systems, such as those found on Netflix and Amazon,
aggregate data across millions of users to make individualized product
suggestions. Each user provides some data about items he or she is
interested in, typically in the form of ratings or clicks, and the
system predicts ratings for unseen items by leveraging data from other
users. Thinking of users as rows and items as columns in a large
matrix, we can formalize these types of problems as \emph{matrix
completion} from a few observed entries. We provide working
implementations of two competing matrix completion algorithms,
Spectral Matrix Completion (SMC) and Singular Value Thresholding
(SVT), and evaluate them on both synthetic and real-world datasets. We
compare their performance against simple baselines, including
Eigentaste, k-nearest neighbors, and item mean. Our comparisons
demonstrate the differences in accuracy and time complexity of these
methods for many problem sizes and structures. All of the presented
methods have their own problem-dependent advantages and disadvantages.
However, in our evaluation on the real-world Jester jokes dataset, we
find that SMC and SVT greatly outperform Eigentaste, an algorithm
specifically designed to leverage special structure in the Jester
dataset. This performance boost likely comes from the fact that SMC
and SVT leverage more data overall when making predictions. Between
SMC and SVT, SMC provides better accuracy but is much slower than SVT.

\end{abstract}

\section{Introduction}

How can we predict what movies a person will enjoy? On what webpages
will they click? Which emails will they read? \emph{Collaborative
  Filtering} problems like these are tricky. Consider predicting a
person's movie preferences using Netflix's database. We can predict
how highly a user will rate the movie \textit{Rambo} using a
regression model trained on the ratings of users that have rated
\textit{Rambo} before. We quickly run into problems when we realize
that most users have not rated \textit{Rambo}, and most users who
have, have not rated the same ones as the user for whom we are making
the prediction. This means that our model would have to ignore a
majority of the data avaialable, limiting performance. Furthermore a
separate model would need to be trained for each movie in the system,
leading to computational difficulties.

A better framework for these sorts of problem is \emph{matrix
  completion}. We can view each user-item pair as an entry in a matrix
that has only a few entries filled in. Our task is then to complete
the matrix, assuming that it has some simple underlying structure. Low
rank is a common structural assumption. Recovering a low rank matrix
from a few entries is a problem with applications not only in
collaborative filtering \cite{r25}, but also in dimensionality
reduction \cite{r20, r28} and multi-task learning \cite{r2, r22}. This
problem turns out to be computationally hard. In fact, rank
minimization is NP-hard, meaning that is at least as hard as all those
that can be solved by a nondeterministic Turing machine in polynomial
time, like the Hamiltonian Cycle and Traveling Salesman problems. In
both theory and practice, all algorithms that perform this task have
time complexity exponential in the size of the matrix. We must settle
for an approximation.

Candes and Recht showed that most low-rank matrices could be recovered
by instead minimizing the nuclear norm, defined as the sum of the
singular values \cite{r4}. Theoretically, this is because the nuclear
norm is the tightest convex lower bound on the rank function for
singular values no greater than 1. Intuitively, while the rank
function counts the number of nonzero singular values, the nuclear
norm sums their amplitude, much like how the 1-norm is a useful
substitute for counting the number of nonzero entries in a vector.
Conveniently, the nuclear norm can be minimized efficiently subject to
equality constraints via semidefinite programming.

Nuclear norm minimization has long been observed to produce fairly
low-rank solutions \cite{r11, r12, r26}), but only recently was there
any theoretical basis for when it produced the true minimum rank
solution. The first paper to provide such foundations was \cite{r24},
where Recht, Fazel, and Parrilo developed probabilistic techniques to
study average case behavior and showed that the nuclear norm heuristic
could solve most instances of the rank minimization problem when the
number of linear constraints was sufficiently large. This inspired a
groundswell of interest in theoretical guarantees for rank
minimization, and these results lay the foundation for \cite{r4}.
Candes and Recht’s bounds were subsequently improved by Candes and Tao
\cite{r7} and Keshavan, Montanari, and Oh \cite{keshavan2010matrix} to
show that one could, in special cases, reconstruct a low-rank matrix
by observing a set of entries of size at most a polylogarithmic factor
larger than the intrinsic dimension of matrix.

To test the validity of these theoretical guarantees in practice, we
explore two competing algorithms for matrix completion via nuclear
norm-minimization and compare their performance on collaborative
filtering baselines, using both synthetic and real-world data. The
Singular Value Thresholding (SVT) algorithm was introduced in ``A
Singular Value Thresholding Algorithm for Matrix Completion''
\cite{cai2010singular}. Spectral Matrix Comletion (SMC) was introduced
in ``Matrix Completion from a Few Entries'' \cite{keshavan2010matrix}
and compare their effectiveness.


\section{Singular Value Thresholding}

Formally, Singular Value Thresholding (SVT) addresses the optimization
problem
\begin{equation*}
\begin{aligned}
  & \underset{X}{\text{minimize}} & & \|X\|_{*} \\
  & \text{subject to}             & & \mathcal{P}_\Omega (X) =
  \mathcal{P}_\Omega (M), \\
\end{aligned}
\end{equation*}
where $\|\cdot\|_{*}$ is the \emph{nuclear norm}, or the sum of the
singular values and $\mathcal{P}_\Omega (\cdot)$ is the projection
operator, which makes zero all entries $(i, j) \notin \Omega$.

Singular Value Thresholding works in an iterative, alternating fashion
reminiscent of the Alternating Direction Method of Multipliers (ADMM).
The complete matrix $X$ is contructed iteratively using a low-rank,
low-singular value approximation to an auxiliary sparse matrix $Y$.
$Y$ is then adjusted to ensure the resulting approximation in the
subsequent step has matching entries $X_{ij} = M_{ij}$. Each iteration
consists of the inductive steps
\begin{equation*}
\begin{cases}
X^{k} = \mathrm{shrink}(Y^{k-1}, \tau) \\
Y^{k} = Y^{k-1} + \delta_k \mathcal{P}_\Omega (M-X^{k}),              \\
\end{cases}
\end{equation*}
where $\mathrm{shrink}(\cdot, \cdot)$ is the \emph{singular value
  shrinkage operator}, which reduces each singular value by aGiven a
singular value decomposition $X = U \Sigma V^T$,
$\Sigma = \mathrm{diag}(\{\sigma_i\}_{1 \le i \le r})$, we can write
this as
\begin{equation*}
\mathrm{shrink}(X, \tau) = U\Sigma_\tau V^T, \ \ \Sigma_\tau = \mathrm{diag}(\{(\sigma_i-\tau)_{+}\}).
\end{equation*}


These two operations, when repeated, approach a low-nuclear norm
solution by repeatedly shrinking the singular values of X. This
algorithm has shown success in recovering accurate low-rank solutions
when the source of $M$ is also low-rank, even though it does not
optimize this objective directly. The original authors discuss its
theoretical guarantees in detail, but we choose to omit them in this
discussion.

SVT has hyperparameters that must be carefully tuned to guarantee
convergence. The shrinkage value $\tau$ must be set high in order for
the algorithm to converge, but not too high that it exceeds the true
singular values. The stepsizes $\delta_k$ are similarly sensitive. The
stepsize can be set dynamically as well, though we choose to maintain
a fixed stepsize throughout. Furthermore, we compute the decomposition
of $Y^K$ in batches, which introduces a new batch size parameter $l$
that effects the speed of convergence. Also important is the
initialization of $Y$, for which the authors provide helpful
strategies. Finally, we use the relative error
$\|\mathcal{P}_{\Omega}(X^k-M)\|_F / \|P_{\Omega} (M)\|$ as a stopping
criterion. We terminate when this drops below a small $\epsilon$.

We find from experimentation that SVT is frustratingly sensitive to
the settings of these parameters. The suggested settings only lead to
convergence for a limited range of matrix sizes and ranks, and
choosing the correct settings often requires unrealistic knowledge of
the underlying matrix structure.


\section{Spectral Matrix Completion}

In the present paper, the Spectral Matrix Completion (SMC) algorithm presented in Keshavan et al.~\cite{keshavan2010matrix} is implemented. Lets say, $M$ is the matrix whose entry $(i,j) \in [m] \times [n]$ corresponding to the rating user $i$ would assign to movie/joke $j$. $M^E$ is the $m \times n$ matrix that contains the revealed entries of $M$, and is filled with 0's in the other positions 
\begin{equation}
M^E_{i,j} = \left \{ \begin{array}{rcl}
	M_{i,j} & \mbox{if  }  (i,j) \in E \\
	 0 & \mbox{otherwise} 
	\end{array} \right.
\end{equation}

As presented in~\cite{keshavan2010matrix}, the SMC algorithm has the following structure:
\begin{enumerate}
\item Trim the $M^E$, and let $\widetilde{M}^E$.
\item Project $\widetilde{M}^E$ to $T_r(\widetilde{M}^E)$.
\item Clean residual errors by minimizing the discrepancy F(X,Y).
\end{enumerate}

\subsection{Trimming}
In the trimming step, zero all columns in $M^E$ with degree larger than $2|E|/n$ and set to zero all rows with degree larger than $2|E|/m$, where $|E|$ is the number of non-zero entries in $M$. Trimming leads to 'throwing out information' which makes the underlying true-rank structure more apparent. This effect becomes even more important when the number
of revealed entries per row/column follows a heavy tail distribution, as for real data.

\subsection{Projection}
In the projection step, compute the singular value decomposition (SVD) of $M^E$ (with $\sigma_1 \ge \sigma_2 \ge .....\ge 0$)
\begin{equation}
M^E = \sum\limits_{i=1}^{min(m,n)} \sigma_ix_iy_i^T
\end{equation}
and, then the matrix $T_r(M^E)$ is $(mn/|E|)\sum\limits_{i=1}^r \sigma_ix_iy_i^T$, obtained by setting to 0 all but the $r$ largest singular values. Note that apart from the rescaling factor $(mn/|E|)$, $T_r(M^E)$ is the orthogonal projection of $M^E$ onto the set of rank-r matrices. 

\subsection{Cleaning}
This is the step where all the magic happens in SMC algorithm. Given $X \in R^{m\times r}$, $Y \in R^{n\times r}$ with $X^TX = m1$ and $Y^TY = n1$, 
\begin{equation}
F(X,Y) = \min_{S \in R^{r \times r}} F(X,Y,S)
\end{equation}
\begin{equation}
F(X,Y,S) = \frac{1}{2} \sum\limits_{(i,j) \in E} (M_{ij} - (XSY^T)_{ij})^2
\end{equation}
The cleaning step consists in writing $T_r(\widetilde{M}^E) = X_0S_0Y_0^T$ and minimizing $F(X, Y)$ locally with initial
condition $X = X_0$, $Y = Y_0$. Note that $F(X, Y)$ is easy to evaluate since it is defined by minimizing the quadratic function
$S \mapsto F(X, Y, S)$ over the low-dimensional matrix $S$. Further it depends on $X$ and $Y$ only through
their column spaces. In geometric terms, $F$ is a function defined over the cartesian product of two Grassmann manifolds. Optimization over Grassmann manifolds is a well understood topic~\cite{edelman1998geometry} and efficient algorithms (in particular Newton and conjugate gradient) can be applied. In the present algorithm, gradient descent with line search is used to minimize $F(X, Y)$.

\section{Baseline Algorithms}

For comparison, we introduce two simple baseline algorithms for matrix
completion.

\subsection{Eigentaste}

Eigentaste is a collaborative filtering algorithm specifically
designed for the Jester Dataset. It relies on the assumption that a
subset of the columns will be dense, corresponding to a few ``seed''
items which every user is required to review. Eigentaste applies
principal component analysis to this dense subset of columns, reducing
dimensionality and allowing rapid clustering. Recommendation is done
by averaging ratings over clusters.

Mean rating of the jth item in the gauge set is given by
\begin{equation*}
\mu_{ij}=\frac{1}{n}{\sum_{i\epsilon U_{j}}}\widetilde{r}_{ij}\\
\end{equation*}
\begin{equation*}
\sigma_j^2=\frac{1}{n-1}{\sum_{i\epsilon U_{j}}}({\widetilde{r}_{ij}-\mu_{j}})^{2}
\end{equation*}

In A, the normalized rating $r_{ij}$ is set to
$({\widetilde{r}_{ij}-\mu_{j}})/\sigma _{j}$ . The global correlation
matrix is given by
\begin{equation*}
C=\frac{1}{n-1}A^{T}A=E^{T} \Lambda E
\end{equation*}

The data is projected along the first v eigenvectors $x=R{E_{v}}^{T}$

TODO: make this next part more concise

\textit{Recursive Rectangular Clustering: }

\begin{enumerate}
\item Find the minimal rectangular cell that encloses all the points
  in the eigenplane.
\item Bisect along x and y axis to form 4 rectangular sub-cells.
\item Bisect the cells in step 2 with origin as a vertex to form
  sub-cells at next hierarchial level.
\end{enumerate}

 \textit{Online Computation of Recommendations}

\begin{enumerate}
\item Collect ratings of all items in gauge set.
\item Use PCA to project this vector to eigenplane.
\item Find the representative cluster.
\item Look up appropriate recommendations, present them to the new
  user, and collect ratings.
\end{enumerate}

\subsection{K-Nearest Neighbors}
In the K-Nearest Neighbors algorithm, we predicts ratings based on the
mean ratings of each person's nearest neighbors.
\begin{equation*}
  p_{ij}=\overline{r_{i}}+\kappa\sum_{k=1}^n w(i,p)(\overline{r_{pj}}-\overline{r_{p}})
\end{equation*}
where $\overline{r_{i}}$ is the average joke rating for user i, and
$\kappa$ is a normalizing factor ensuring that the absolute value of
the weights sum to 1. We implemented the weighted nearest neighbor
algorithm. We used a function of Euclidean distance from user i to
user p as the weight w(i, p), and $\kappa$ = $\sum_{k=1}^n w(i,p)$.
Specifically, if we are interested in q nearest neighbors, $w(i; p) =
d(i, q +1)- d(i, p)$. This ensures that i’s closest neighbor has the
largest weight.

\section{Experimental Results}

\subsection{Synthetic Data}

The performance of the SVT and SMC algorithms were compared on
synthetically generated matrices of varying ranks, sizes and sparsity.
Specifically, the algorithms were tested on all combinations of matrix
size(1000, 5000), rank (5, 10 and 20), and sparsity(30\%, 50\% and
70\%). As a baseline to SVT and SMC, a column mean matrix completion
method was implemented. This method predicts the average column value
for all missing entries. In the Tables~\ref{MAE},~\ref{Time} the mean
absolute error and the run time is averaged over the categories of
size, rank and sparsity. For example, out of the eighteen tests that
were run, nine tests include the matrix of size 1000x1000. Therefore
the mean absolute error value for all nine tests are averaged in order
to represent the values in the size 1000 column of the table below.
Mean absolute error is the absolute value of the average error on the
unknown entries. Run time is measured in seconds and all simulations
were run at University of Michigan's CAEN lab computers (8GB memory
with one core i7 processors). The columns in
Tables~\ref{MAE},~\ref{Time} contain the averaged mean absolute error
and the run time for all experiments.



% I ran this experiment but the results were pretty boring. We can
% choose include them if we want (Just uncomment) but it looks like we
% already have plenty of content. I also set this up to visualize the
% steps of SVT but they also were pretty uninteresting.



%\subsubsection{Low-Rank Recovery}

%Both Singular Value Thresholding and Spectral Matrix Completion
%actively seek low-rank solutions. Realistically, the matrices we hope
%to complete are not truly low-rank, but can often be approximated as
%such. We investigate both algorithms' ability to discover low-rank
%structure from noisy matrices in table \ref{rankrecover}.

%\begin{table} [ht!]
% \label{rankrecover}
%\centering
% \begin{tabular}{c|c c c c c} % p{2cm}}
%   \hline \hline
%                & \multicolumn{5}{c}{Noise level} \\
%   Algorithm    & 0.0 & 0.1 & 1.0 & 5.0 & 10.0 \\ \hline
%   SVT          & 5 & 6 & 5 & 6 & 4 \\
%   SMC          & 5 & 5 & 5 & 5 & 3\\
%   \hline \hline
% \end{tabular}
% \caption{\footnotesize Rank recovered from noisy entries from rank-5 matrices. 
% Matrices are size $100 \times 100$ with 25\% sparsity. Gaussian
% noise with increasing variance is added to each known entry. Noise
% level is the standard deviation of this noise}
%\end{table}

%We find that both methods are able to accurately recover the low-rank
%structure of matrices from a small number of noisy entries. It should
%however be noted that Singular Value Thresholding relies on heavy
%hyper-parameter tuning and may not behave as reliably when the true
%rank is unknown.


While low rank is a desirable property and a suitable target for
optimization, we are primarily concerend with accurately predicting
unknown entries in a matrix. 

\begin{table} [ht!]
\begin{center}
 \begin{tabular}{l | l l l | l l l |l l}% p{2cm}}
  \hline \hline
               & \mc{3}{c}{Rank}             & \mc{3}{c}{Sparsity}         & \mc{2}{c}{Size ($n \times n$)} \\
  Algorithm    & 5 & 10 & 20                 & 30\% & 50\% & 70\%          & 1000 & 5000 \\ \hline
  Mean         & 5.65 & 9.90 & 13.98         & 9.10 & 10.21 & 10.21        & 9.43 & 10.26\\
  SVT          & 2.05 & 7.2E-04 & 1.4E-03    & 2.05 & 8.4E-04 & 9.6E-04    & 1.37 & 7.6E-04\\
  SMC          & 2.1E-04 & 4.3E-07 & 4.5E-07 & 2.1E-04 & 4.2E-07 & 3.8E-07 & 1.4E-04 & 3.7E-07\\
 \hline \hline
 \end{tabular}
 \caption{Mean Absolute Error of predictions of unknown entries. TODO:
 add more details about experiment parameters}
\end{center}
\label{MAE}
\end{table}

Table \ref{MAE} shows that both SVT and SMC outperform the baseline
column-mean method, which simply fills in all unknown entries in a
column with the mean of its known entries. In addition, SMC
consistently outperforms SVT in terms of mean absolute error, often by
several orders of magnitude.

We are also concerned with the speed at which we arrive at accurate
predictions. Table \ref{Time} shows that, while SMC significantly
outperforms SVT in terms of accuracy, SVT significantly outperforms
SMC in terms of time. It appears that the computational cost of SMC
grows exponentially with the rank. Even at rank $5$, the smallest
tested rank value, SMC runs 9.6 times slower than SVT.


\begin{table} [ht!]
\centering
 \begin{tabular}{l | l l l |l l l | l l}% p{2cm}}
  \hline \hline
             &  \mc{3}{c}{Rank}   & \mc{3}{c}{Sparsity} & \mc{2}{c}{Size ($n \times n$)} \\
  Algorithm  & 5 & 10 & 20        & 30\% & 50\% & 70\%  & 1000 & 5000 \\ \hline
  Mean       & 0.12 & 0.12 & 0.10 & 0.11 & 0.11 & 0.12  & 0.01 & 0.22\\
  SVT        & 26 & 39 & 92       & 42 & 54 & 60        & 6 & 98 \\
  SMC        & 252 & 405 & 1623   & 1021 & 510 & 748    & 86 & 1433 \\
  \hline \hline
 \end{tabular}
 \caption{Run Time (in seconds) TODO: add more details about
 experiment parameters}
 \label{Time}
\end{table}



\subsection{Jester Dataset}

The Jester Joke dataset contains 4.1 million ratings for 100 jokes
from 73,421 users \cite{r30}. A set of 10 ``seed'' jokes were chosen
to be presented to users before any others, and users that did not
rate all of the seed jokes were discarded. This leaves us with 10
completely dense columns, allowing us to apply both supervised
learning algorithms (Eigentaste) and matrix completion algorithms
(SVT, SMC). We hypothesize that the matrix completion algorithms will
outperform those that only take advantage of the dense columns because
they are capable of utilizing all data during training.

For the purpose of evaluation we randomly select subsets of 100, 200,
and 1000 users' ratings from the Jester Dataset. We choose two ratings
at random from each user as test points and leave the remainder for
training. We evaluate three algorithms, Eigentaste, SMC, and SVT,
using the Normalized Mean Absolute Error (NMAE) of the reconstruction
on the test points, similar to \cite{oh2010thesis}.

The NMAE, a commonly used performance metric in collaborative
filtering, is defined in terms of The Mean Absolute Error (MAE):
\begin{equation}
MAE = \frac{1}{|T|}\sum\limits_{(u,i) \in T} |M_{ui} - \widetilde{M}_{ui}|
\end{equation}
where $M_{ui}$ is the original ratings, $\widetilde{M}_{ui}$ is the predicted rating for user $u$, item $i$, and $T$ is the test set. The NMAE can be defined as:
\begin{equation}
NMAE = \frac{MAE}{M_{max}-M_{min}}
\end{equation}
where $M_{max}$ and $M_{min}$ are the upper and lower bounds for the
ratings. In the Jester joke dataset, the rating are in the range
$[-10,10]$.

%\begin{figure}[h!]
%  \centering
%\includegraphics[trim=0 0 0 0, clip, width=1.0\linewidth]{eigentaste_jester.tiff}
%\caption{\footnotesize Figure shows the comparison between the
% original NACA 0009, undeformed multipoint hydrostructural optimized
% foil, and deformed multipoint hydrostructural optimized foil at
% $C_L = 0.65$. Figure also shows the zoomed-in view of the trailing
% edge thickness at the root. Note the increased thickness for the
% multipoint optimized foil at the root to meet the stress constraints
% for the higher loading cases. The trailing thickness of the
% optimized foil is higher than the original NACA 0009 foil, can also
% be noted. The results are obtained using the RANS solver with
% $Re= 1.0 \times 10^6$ and $V = 12.4$ m/s. }
%\label{Geo}
%\end{figure}

In the Table~\ref{SMC}, we present numerical results on the Jester
joke dataset with the SMC algorithm.

\begin{table} [h]
\centering
 \begin{tabular}{l l l l l}% p{2cm}}
   \hline \hline
   \# users & \# jokes & samp. ratio & NMAE & Time (s)\\
   \hline
   100 & 100 & 5353 & 0.1573 & 25.31\\
   200 & 100 & 10921 & 0.1603 & 17.44\\
   1000 & 100 & 57578 & 0.1647 & 44.95\\
   \hline \hline
 \end{tabular}
 \caption{Numerical results on the Jester joke dataset with SMC
   algorithm. Times reported are from a University of Michigan's CAEN
   lab computer(8GB memory with one core i7 processor)}
\label{SMC}
\end{table}


\begin{table} [h]
\centering
 
 \begin{tabular}{l l | l l l }% p{2cm}}
  \hline \hline
   & & \mc{3}{c}{NMAE}\\
  \# user & \# jokes & SMC  & SVT & Eigentaste\\
\hline
100 & 100 & 0.1573 & 0.1865 &0.187\\
200 & 100 & 0.1603 & 0.1843 & 0.190\\
1000 & 100 & 0.1647& 0.1714 & 0.237\\
\hline \hline
\end{tabular}
 \caption{NMAE comparison on the Jester joke dataset for SMC, SVT,
 and Eigentaste algorithms.}
\label{Compare}
 
\end{table}

We have eigentaste, SMC and SVT on subsets of the Jester dataset.
Compare accuracy and time. Mention that we used LMSVD in SMC for this
version.

\subsubsection{Visualization}

Each of the algorithms compared previously rely on the assumption that
the Jester Dataset can approximated using a low-rank matrix. To test
this claim, we plot each of the 100 jester jokes in a two-dimensional
plane and attempt to explain the meaning of the directions. In axes
according to the top two eigenvectors of the Singular Value
Thresholding solution.

\begin{figure} [h]
\centering
\includegraphics[width = 0.75\textwidth]{jokes_2d_2.pdf}
\begin{tabular}{c | l}
\hline \hline
Joke \# & Joke Text \\ \hline
  8     & Q. Did you hear about 
the dyslexic devil worshiper? \\
        & A. He sold his soul to Santa.\\ \hline
  26 & A guy walks into a bar and sits down next to an extremely
       gorgeous woman. The first \\
     & thing he notices about her though, are her pants. They were
       skin-tight, high-waisted \\
     & and had no obvious mechanism (zipper, buttons or velcro) for opening them. \\
     & After several minutes of puzzling over how she got the pants up over
       her hips, he finally \\
     & worked up the nerve to ask her. "Excuse me miss, but how do you get into your pants?" \\ 
     & "Well," she replied, "you can start by buying me a drink."\\ \hline
  29 & An old Scotsmen is sitting with a younger Scottish gentleman
       and says the \\
     &  boy. "Ah, lad look out that window. You see that stone wall
       there, I built it \\
     & with me own bare hands, placed every stone meself. But do they
       call me MacGregor \\
     & the wall builder? No! \\
     & He Takes a few sips of his beer then says, "Aye, and look out
       on that lake and \\
     & eye that beautiful pier. I built it meself, laid every board
       and hammered each nail \\
     & but do they call me MacGregor the pier builder? No! \\
     & He continues..."And lad, you see that road? That too I build with me
       own bare hands. \\
     & Laid every inch of pavement meself, but do they
       call MacGregor the road builder? No!" \\
     & Again he returns to his beer for a few sips, then says,
       "Agh, but you screw  one sheep..." \\ \hline
  18 & A dog walks into Western Union and asks the clerk to send a
       telegram. He fills out a \\
     & form on which he writes down the telegram he wishes to send:
       "Bow wow wow, Bow wow wow." \\
     & The clerk says, "You can add another 'Bow wow' for the same
       price."\\
     & The dog responded, "Now wouldn't that sound a little silly?"\\ 
\hline \hline
\end{tabular}

\caption{Two-dimensional projection of the Jester joke dataset. The
  horizontal and vertical axes correspond to the directions of the
  first and second eigenvectors, respectively, of the completed matrix
  found by the SVT algorithm on the full Jester dataset. Also given
  are the most extreme jokes in each direction, which characterize the
  axes.}
\label{jokes2d}
\end{figure}

In figure \ref{jokes2d} we find that the two principal components in
the space of jokes correspond to short, simple puns (left) versus
longer, complex jokes (right), and rowdy jokes (top) versus clean
jokes (bottom). Only a few jokes are given here, but at closer
inspection these observations seem to hold true in general.


\section{Conclusions}

General Points: synthetic data conclusions: SMC and SVT are better
than baseline mean method. SMC attains better accuracy than SVT but
takes longer. One potential improvement to SMC is to use the LMSVD
mentioned in the SVT paper. We used this upgrade in the Jester dataset
tests and we found that



\section{Group Member's Accomplishments}

All team members participated in data processing early in the project.
All team members contributed to the final report. Each member was
responsible for the writeup regarding their individual experiments.
Sriram spearheaded the implementation of Eigentaste. He discovered
ways of modifying these methods to supercharge performance and also
contributed a great deal towards the performance evaluation for
different algorithms. Nitin and Jeeheh implemented the SMC algorithm.
Jeeheh took charge in getting the performance comparison for the
synthetic data. Nitin contributed towards the performance metric for
different algorithms and also helped integrating the final report
together. Jonathan was in command of implementing the SVT algorithm
and tuning the parameters for experiments. He also created the
nice-looking visualizations of the Jester joke in the report.


\bibliographystyle{plain}
\bibliography{writeup}

\end{document}
