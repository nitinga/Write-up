
\documentclass{article} % For LaTeX2e
\usepackage{nips15submit_e,times}
\usepackage{hyperref}
\usepackage{url}
\usepackage{amsmath}
\usepackage{graphicx}
%\documentstyle[nips14submit_09,times,art10]{article} % For LaTeX 2.09


\title{Singular Value Thresholding for Matrix Completion}


\author{
Sriram Ganesan\thanks{ The names are printed in alphabetical order by last name.} \\
\texttt{sriramg@umich.edu} \\
\And
Nitin Garg \\
\texttt{gargn@umich.edu} \\
\AND
Jeeheh Oh \\
\texttt{jeeheh@umich.edu} \\
\And
Jonathan Stroud \\
\texttt{stroud@umich.edu} \\
}

% The \author macro works with any number of authors. There are two commands
% used to separate the names and addresses of multiple authors: \And and \AND.
%
% Using \And between authors leaves it to \LaTeX{} to determine where to break
% the lines. Using \AND forces a linebreak at that point. So, if \LaTeX{}
% puts 3 of 4 authors names on the first line, and the last on the second
% line, try using \AND instead of \And before the third author name.

\newcommand{\fix}{\marginpar{FIX}}
\newcommand{\new}{\marginpar{NEW}}

\nipsfinalcopy % Uncomment for camera-ready version

\begin{document}


\maketitle

\begin{abstract}
%Recommender systems, such as those found on Netflix and Amazon,
aggregate data across millions of users to make individualized product
suggestions. Each user provides some data about items he or she is
interested in, typically in the form of ratings or clicks, and the
system predicts ratings for unseen items by leveraging data from other
users. Thinking of users as rows and items as columns in a large
matrix, we can formalize these types of problems as \emph{matrix
completion} from a few observed entries. We provide working
implementations of two competing matrix completion algorithms,
Spectral Matrix Completion (SMC) and Singular Value Thresholding
(SVT), and evaluate them on both synthetic and real-world datasets. We
compare their performance against simple baselines, including
Eigentaste, k-nearest neighbors, and item mean. Our comparisons
demonstrate the differences in accuracy and time complexity of these
methods for many problem sizes and structures. All of the presented
methods have their own problem-dependent advantages and disadvantages.
However, in our evaluation on the real-world Jester jokes dataset, we
find that SMC and SVT greatly outperform Eigentaste, an algorithm
specifically designed to leverage special structure in the Jester
dataset. This performance boost likely comes from the fact that SMC
and SVT leverage more data overall when making predictions. Between
SMC and SVT, SMC provides better accuracy but is much slower than SVT.

\end{abstract}

\section{Introduction}

\section{Different Algorithms}

\subsection{Baseline Algorithm (Eigentaste)}

Eigentaste is a collaborative filtering algorithm which applies principal component analysis to the dense subset of user ratings, thus facilitating dimensionality reduction for offline clusters and rapid computation of recommendations. 
Mean rating of the jth item in the gauge set is given by
\begin{equation*}
\mu_{ij}=\frac{1}{n}{\sum_{i\epsilon U_{j}}}\widetilde{r}_{ij}\\
\end{equation*}
\begin{equation*}
\sigma_j^2=\frac{1}{n-1}{\sum_{i\epsilon U_{j}}}({\widetilde{r}_{ij}-\mu_{j}})^{2}
\end{equation*}

In A, the normalized rating $r_{ij}$ is set to $({\widetilde{r}_{ij}-\mu_{j}})/\sigma _{j}$ . The global correlation matrix is given by
\begin{equation*}
C=\frac{1}{n-1}A^{T}A=E^{T} \Lambda E
\end{equation*}

The data is projected along the first v eigenvectors x=R${E_{v}}^{T}$

\textit{Recursive Rectangular Clustering: }

1). Find the minimal rectangular cell that encloses all the points in the eigenplane. \\
2) Bisect along x and y axis to form 4 rectangular sub-cells.\\
3) Bisect the cells in step 2 with origin as a vertex to form sub-cells at next hierarchial level.

 \textit{Online Computation of Recommendations}

1) Collect ratings of all items in gauge set.\\
2) Use PCA to project this vector to eigenplane.\\
3) Find the representative cluster.\\
4) Look up appropriate recommendations, present them to the new user, and collect ratings.




\subsection{SMC}
%In the present paper, the Spectral Matrix Completion (SMC) algorithm presented in Keshavan et al.~\cite{keshavan2010matrix} is implemented. Lets say, $M$ is the matrix whose entry $(i,j) \in [m] \times [n]$ corresponding to the rating user $i$ would assign to movie/joke $j$. $M^E$ is the $m \times n$ matrix that contains the revealed entries of $M$, and is filled with 0's in the other positions 
\begin{equation}
M^E_{i,j} = \left \{ \begin{array}{rcl}
	M_{i,j} & \mbox{if  }  (i,j) \in E \\
	 0 & \mbox{otherwise} 
	\end{array} \right.
\end{equation}

As presented in~\cite{keshavan2010matrix}, the SMC algorithm has the following structure:
\begin{enumerate}
\item Trim the $M^E$, and let $\widetilde{M}^E$.
\item Project $\widetilde{M}^E$ to $T_r(\widetilde{M}^E)$.
\item Clean residual errors by minimizing the discrepancy F(X,Y).
\end{enumerate}

\subsection{Trimming}
In the trimming step, zero all columns in $M^E$ with degree larger than $2|E|/n$ and set to zero all rows with degree larger than $2|E|/m$, where $|E|$ is the number of non-zero entries in $M$. Trimming leads to 'throwing out information' which makes the underlying true-rank structure more apparent. This effect becomes even more important when the number
of revealed entries per row/column follows a heavy tail distribution, as for real data.

\subsection{Projection}
In the projection step, compute the singular value decomposition (SVD) of $M^E$ (with $\sigma_1 \ge \sigma_2 \ge .....\ge 0$)
\begin{equation}
M^E = \sum\limits_{i=1}^{min(m,n)} \sigma_ix_iy_i^T
\end{equation}
and, then the matrix $T_r(M^E)$ is $(mn/|E|)\sum\limits_{i=1}^r \sigma_ix_iy_i^T$, obtained by setting to 0 all but the $r$ largest singular values. Note that apart from the rescaling factor $(mn/|E|)$, $T_r(M^E)$ is the orthogonal projection of $M^E$ onto the set of rank-r matrices. 

\subsection{Cleaning}
This is the step where all the magic happens in SMC algorithm. Given $X \in R^{m\times r}$, $Y \in R^{n\times r}$ with $X^TX = m1$ and $Y^TY = n1$, 
\begin{equation}
F(X,Y) = \min_{S \in R^{r \times r}} F(X,Y,S)
\end{equation}
\begin{equation}
F(X,Y,S) = \frac{1}{2} \sum\limits_{(i,j) \in E} (M_{ij} - (XSY^T)_{ij})^2
\end{equation}
The cleaning step consists in writing $T_r(\widetilde{M}^E) = X_0S_0Y_0^T$ and minimizing $F(X, Y)$ locally with initial
condition $X = X_0$, $Y = Y_0$. Note that $F(X, Y)$ is easy to evaluate since it is defined by minimizing the quadratic function
$S \mapsto F(X, Y, S)$ over the low-dimensional matrix $S$. Further it depends on $X$ and $Y$ only through
their column spaces. In geometric terms, $F$ is a function defined over the cartesian product of two Grassmann manifolds. Optimization over Grassmann manifolds is a well understood topic~\cite{edelman1998geometry} and efficient algorithms (in particular Newton and conjugate gradient) can be applied. In the present algorithm, gradient descent with line search is used to minimize $F(X, Y)$.

\subsection{Singular Value Thresholding}

Singular Value Thresholding (SVT) \cite{cai2010singular} is an
algorithm proposed for \emph{nuclear norm minimization} of a matrix
$X$ from a few known entries $M_{ij}, (i,j) \in \Omega$. Formally, SVT
addresses the optimization problem

\begin{equation*}
\begin{aligned}
  & \underset{X}{\text{minimize}} & & \|X\|_{*} \\
  & \text{subject to}             & & \mathcal{P}_\Omega (X) =
  \mathcal{P}_\Omega (M), \\
\end{aligned}
\end{equation*}

where $\|\cdot\|_{*}$ is the \emph{nuclear norm}, or the sum of the
singular values and $\mathcal{P}_\Omega (\cdot)$ makes zero all
entries $(i, j) \notin \Omega$. This can be thought of as a convex
relaxation to the rank minimization problem, and the two are formally
equivalent under some conditions. The rank minimization problem is,
however, highly non-convex and therefore not a suitable candidate for
black-box optimization algorithms.

Singular Value Thresholding works by iteratively constructing $X$
using a low-rank, low-singular value approximation to an auxiliary
sparse matrix $Y$. $Y$ is then adjusted to ensure the resulting
approximation in the subsequent step has matching entries
$X_{ij} = M_{ij}$. Each iteration consists of the inductive steps

\begin{equation*}
\begin{cases}
X^{k} = \mathrm{shrink}(Y^{k-1}, \tau) \\
Y^{k} = Y^{k-1} + \delta_k \mathcal{P}_\Omega (M-X^{k}),              \\
\end{cases}
\end{equation*}

where $\mathrm{shrink}(\cdot, \cdot)$ is the \emph{singular value
  shrinkage operator}. Given a singular value decomposition $X = U
\Sigma V^T$, $\Sigma = \mathrm{diag}(\{\sigma_i\}_{1 \le i \le r})$, we
  can write this as

\begin{equation*}
\mathrm{shrink}(X, \tau) = U\Sigma_\tau V^T, \ \ \Sigma_\tau = \mathrm{diag}(\{(\sigma_i-\tau)_{+}\}).
\end{equation*} 

These two operations, when repeated, approach a low-nuclear norm
solution by repeatedly shrinking the singular values of X. This
algorithm has shown success in recovering accurate low-rank solutions
when the source of $M$ is also low-rank, even though it does not
optimize this objective directly. The original authors discuss its
theoretical guarantees in detail, but we choose to omit them in this
discussion.

In practice, this system has a number of hyperparameters that must be
carefully tuned to guarantee convergence. The shrinkage value $\tau$
must be set fairly high in order for the algorithm to converge
quickly, but not too high that it dwarfs the true singular values. The
stepsizes $\delta_k$ are similarly sensitive. These can be set
dynamically as well, though we choose to maintain a fixed stepsize
throughout. We compute the decomposition of $Y^K$ in batches, which
introduces a new batch size parameter $l$. Also important is the
initialization of $Y^0$, for which the authors provide helpful
strategies. Finally, we use the relative error

$\|\mathcal{P}_{\Omega}(X^k-M)\|_F / \|P_{\Omega} (M)\|$ as a stopping
criterion; we terminate when this drops below a small $\epsilon$.



\section{Results and Discussion}
\subsection{Synthetic Data}

%\begin{table}[H!]
\centering
\begin{tabular}{|c|c|c|c|c|c|c|}
\hline
size ($n \times n$) & rank ($r$) & $m/d_r$ & $m/n^2$ & time (s) & \#iters & relative error \\
\hline
1000 & 10 & 6 & 0.119 & 279.1 & 250.0 & 86.59246 x $10^{-4}$ \\
\hline
1000 & 50 & 4 & 0.390 & 800.4 & 220.4 & 0.99209 x $10^{-4}$ \\
\hline
1000 & 100 & 3 & 0.570 & 604.6 & 163.8 & 0.98189 x $10^{-4}$ \\
\hline
5000 & 10 & 6 & 0.024 & 12934.9 & 250.0 & 611.42446 x $10^{-4}$ \\
\hline
5000 & 50 & 5 & 0.100 & - & - & - \\
\hline
5000 & 100 & 4 & 0.158 & - & - & - \\
\hline
10000 & 10 & 6 & 0.012 & - & - & - \\
\hline
10000 & 50 & 5 & 0.050 & - & - & - \\
\hline
10000 & 100 & 4 & 0.080 & - & - & - \\
\hline
20000 & 10 & 6 & 0.006 & - & - & - \\
\hline
20000 & 50 & 5 & 0.025 & - & - & - \\
\hline
30000 & 10 & 6 & 0.004 & - & - & - \\
\hline
\end{tabular}
\caption{Performance of Singular Value Thresholding on synthetic matrices of known rank. We generate two $n\times r$ matrices $U$ and $V$ whose entries are i.i.d. gaussian.We choose $m$ random entries from $M = U V^T$ and measure convergence rates of SVT. $m/d_r$ is the ratio of sampled entries $m$ and the ``true dimensionality'' $d_r = r(2n-r)$.}
\label{table:svt_synth_1}
\end{table}


\subsection{Jester Dataset}

\section{Conclusions}

\section{Future Work}


\bibliographystyle{plain}
\bibliography{writeup}

\end{document}
